\chapter{Kode Program \textit{Route}}
\label{lamp:F}

\begin{lstlisting}[language=tex,basicstyle=
\tiny,caption=route.php]
<?php
/*
|--------------------------------------------------------------------------
| Web Routes
|--------------------------------------------------------------------------
|
| Here is where you can register web routes for your application. These
| routes are loaded by the RouteServiceProvider within a group which
| contains the "web" middleware group. Now create something great!
|
*/

Route::get('/reg', function(){
  return view('auth/register');
});
Route::post('/register', 'Auth\RegisterController@register');
Route::group(['prefix' => 'Api'], function(){
    Route::get('/showFormatSurat', 'Api\FormatsuratAPIController@tampilkanFormat');
});
Route::get('/', function () {
    return view('welcome');
});

Route::get('/laravel-login', function(){
  return view('auth.login');
});

Route::get('/logout', 'AuthController@logout');
Route::post('/home', 'AuthController@authenticate');
//<-------------------------------------------------------MAHASISWA------------------------------------------------------->
    // halaman utama mahasiswa
    Route::get('/home_mahasiswa', 'HistorysuratController@tampilkanProfil');
    //halaman untuk memilih jenis surat
    Route::get('/pilih_kategori_surat', 'MahasiswaController@kategoriSurat');
    Route::post('/pilih_jenis_surat', 'FormatsuratController@pilihSurat');
    //halaman untuk pengisian data diri untuk masing-masing surat
    Route::post('/isi_data_diri', 'FormatsuratController@tampilkanFormulir');
    //halaman untuk menampilkan preview surat
    Route::post('/preview', 'PesanansuratController@tampilkanPreview');
    Route::post('/kirimFormulir', 'PesanansuratController@store');
//<-------------------------------------------------------PETUGAS TU------------------------------------------------------>
    // halaman utama pejabat
    Route::get('/home_TU', 'PesanansuratController@tampilkanPesananSurat');
    //halaman seluruh format surat
    Route::get('/format_surat', 'FormatsuratController@tampilkanSeluruhFormat');
    //untuk menghapus mahasiswa dari database
    Route::post('/hapusMahasiswa', 'MahasiswaController@destroy');
    //untuk menghapus format surat dari database
    Route::post('/hapusFormatsurat', 'FormatsuratController@destroy');
    Route::post('/tampilkanFormat', 'FormatsuratController@tampilkanFormat');
    Route::post('/proses_surat', 'PesanansuratController@sendDataSurat');
    Route::get('/history_TU', 'HistorysuratController@pilihHistorySurat');
    // halaman untuk menambahkan format surat baru
    Route::get('/tambah_format_surat', 'FormatsuratController@tambahFormat');
    Route::get('/tambah_data_mahasiswa', 'MahasiswaController@tambahDataMahasiswa');
    Route::post('/uploadFormat', 'FormatsuratController@storeFormat');
    //halaman seluruh data mahasiswa
    Route::get('/data_mahasiswa', 'MahasiswaController@pilihMahasiswa');
    // halaman untuk menambahkan data mahasiswa
    Route::post('/uploadDataMhs', 'MahasiswaController@uploadMahasiswa');
    Route::post('/generatePDF', 'HistorysuratController@buatPDF');
    Route::get('/setting', 'MahasiswaController@setting');
    Route::post('/updateSemester', 'MahasiswaController@updateSemester');
//<--------------------------------------------------------PEJABAT-------------------------------------------------------->
    // halaman utama pejabat
    Route::get('/home_pejabat', 'PesanansuratController@tampilkanPesananDiPejabat');
    Route::post('/persetujuan', 'PesanansuratController@tambahPersetujuan');
    Route::post('/previewCatatan', 'PesanansuratController@previewDosen');
    Route::post('/updateCatatan', 'PesanansuratController@updateCatatan');
    Route::get('/history_pejabat', 'HistorysuratController@tampilkanHistoryDiPejabat');
    Route::post('/ubahStatusPenandatanganan', 'HistorysuratController@ubahStatusPenandatanganan');
    Route::post('/ubahStatusPengambilan', 'HistorysuratController@ubahStatusPengambilan');
    Route::post('/updateFormulir','PesanansuratController@updateFormulir');

Auth::routes();

Route::get('/home', 'HomeController@index');
\end{lstlisting}