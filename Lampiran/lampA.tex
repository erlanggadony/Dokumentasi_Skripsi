%versi 2 (8-10-2016)
\chapter{Kode Program \textit{View}}
\label{lamp:A}

%selalu gunakan single spacing untuk source code !!!!!
\singlespacing 
% language: bahasa dari kode program
% terdapat beberapa pilihan : Java, C, C++, PHP, Matlab, R, dll
%
% basicstyle : ukuran font untuk kode program
% terdapat beberapa pilihan : tiny, scriptsize, footnotesize, dll
%
% caption : nama yang akan ditampilkan di dokumen akhir, lihat contoh

\begin{lstlisting}[language=php, basicstyle=\tiny, caption=\textit{Home} mahasiswa]
	<!DOCTYPE html>
  <head>
      <title>Home - Mahasiswa</title>
      <link href="{{ asset("/bootstrap-3.3.7-dist/css/bootstrap.css") }}" rel="stylesheet" type="text/css" />
      <link href="{{ asset("/css/styles_list_surat.css") }}" rel="stylesheet" type="text/css">

  </head>

  <body>
    <div>
        <img id=banner src="{{ asset("/images/banner ftis.png") }}" />
    </div>

    <!-- Navigation here -->
    @include('mahasiswa.menu')

    <div class="container">
      <div class="main">
          <div class="row">
            <div class="col-md-8 content">
              <form class="form-inline" action= "{{ url('/data_mahasiswa') }}" method="get">
                <div class="form-group">
                  <label for="kategori_mahasiswa">Cari berdasarkan :</label><br>
                  <select name="kategori" class="form-control">
                    <option value="tanggalBuat">Cari semua surat</option>
                    <option value="tanggalBuat">Tanggal pembuatan</option>
                    <option value="perihal">Perihal</option>
                    <option value="kepada">Kepada</option>
                    <option value="jenis_surat">Jenis surat</option>
                  </select>
                </div>
                <div class="form-group">
                  <label for="searchBox">Kata kunci :</label><br>
                  <input type="text" name="searchBox" class="form-control" size="80" />
                  <button type="submit" name="findmail" class="btn btn-primary">Cari surat</button>
                </div>
              </form>
              <br>
              <table class="table table-striped">
                @if($historysurats != null)
                  @if(count($historysurats) == 0)
                      <tr>
                          <td colspan="5" align="center">Tidak ada history surat....</td>
                      </tr>
                  @else
                      <tr>
                        <th>TANGGAL PEMBUATAN</th>
                        <th>PERIHAL</th>
                        <th>KEPADA</th>
                        <th>JENIS SURAT</th>
                      </tr>
                      @foreach($historysurats as $historysurat)
                        <tr>
                          <td class="ctr">{{ $historysurat->created_at }}</td>
                          <td class="ctr">{{ $historysurat->perihal }}</td>
                          <td class="ctr">{{ $historysurat->penerimaSurat }}</td>
                          <td class="ctr">{{ $historysurat->formatsurat->jenis_surat }}</td>
                        </tr>
                      @endforeach
                  @endif
                @endif
              </table>
            </div>
            @include('mahasiswa.profile_bar')
          </div>
      </div>
    </div>
    <div class="footer">
    </div>
  </body>
</html>

\end{lstlisting}


\begin{lstlisting}[language=Java,basicstyle=\tiny,caption=Pilih kategori surat]
	
<!DOCTYPE html>
  <head>
      <title>Home</title>
      <link href="{{ asset("/bootstrap-3.3.7-dist/css/bootstrap.css") }}" rel="stylesheet" type="text/css" />
      <link href="{{ asset("/css/styles_list_surat.css") }}" rel="stylesheet" type="text/css">

  </head>

  <body>
    <div>
        <img id=banner src="{{ asset("/images/banner ftis.png") }}" />
    </div>


    <!-- Navigation Here -->
    @include('mahasiswa.menu')

    <div class="container">
      <div class="main">
        <div class="row">
          <div class="col-md-8 content">
            <h1>Pilih Kategori Surat</h1>
            <br>
            <form class="form-horizontal" action="{{ url('/pilih_jenis_surat') }}" method="post">
              <div class="form-group">
                <div class="col-sm-9">
                    <div class="radio">
                      <label>
                        <input type="radio"  name="jenis_surat" value="surat_izin" required>
                        Surat Izin
                      </label>
                    </div>
                    <div class="radio">
                      <label>
                        <input type="radio"  name="jenis_surat" value="surat_keterangan" required>
                        Surat Keterangan
                      </label>
                    </div>
                    <div class="radio">
                      <label>
                        <input type="radio"  name="jenis_surat" value="surat_perwakilan" required>
                        Surat Perwakilan
                      </label>
                    </div>
                    <div class="radio">
                      <label>
                        <input type="radio"  name="jenis_surat" value="surat_pengantar" required>
                        Surat Pengantar
                      </label>
                    </div>
                </div>
              </div>
              {!! csrf_field() !!}
              <div class="form-group">
                <div class="col-sm-6">
                  <button type="submit" class="btn btn-primary">Lanjutkan</button>
                </div>
              </div>
            </form>
          </div>
          @include('mahasiswa.profile_bar')
        </div>
      </div>
    </div>
    <div class="footer">
    </div>
  </body>
</html>

\end{lstlisting}

\begin{lstlisting}[language=Java,basicstyle=\tiny,caption=Pilih jenis surat keterangan]
	<!DOCTYPE html>
  <head>
      <title>Pilih Jenis Surat</title>
      <link href="{{ asset("/bootstrap-3.3.7-dist/css/bootstrap.css") }}" rel="stylesheet" type="text/css" />
      <link href="{{ asset("/css/styles_list_surat.css") }}" rel="stylesheet" type="text/css">

  </head>

  <body>
    <div>
        <img id=banner src="{{ asset("/images/banner ftis.png") }}" />
    </div>


    <!-- Navigation Here -->
    @include('mahasiswa.menu')

    <div class="container">
      <div class="main">
        <div class="row">
          <div class="col-md-8 content">
            <h1>Pilih Jenis Surat</h1>
            <br>
            <form class="form-horizontal" action="{{ url('/isi_data_diri') }}" method="post">
              <div class="form-group">
                <div class="col-sm-9">
                    @foreach($formatsurats as $formatsurat)
                      @if(($formatsurat->id == 1) || ($formatsurat->id == 2))
                        <div class="radio">
                          <label>
                            <input type="radio"  name="jenis_surat" value="{{ $formatsurat->id }}" required>
                            {{ $formatsurat->jenis_surat }}
                          </label>
                        </div>
                      @endif
                    @endforeach
                </div>
              </div>
              {!! csrf_field() !!}
              <div class="form-group">
                <div class="col-sm-6">
                  <button type="submit" class="btn btn-primary">Lanjutkan</button>
                </div>
              </div>
            </form>
          </div>
          @include('mahasiswa.profile_bar')
        </div>
      </div>
    </div>
    <div class="footer">
    </div>
  </body>
</html>

\end{lstlisting}

\begin{lstlisting}[language=Java,basicstyle=\tiny,caption=Pengisian data keterangan beasiswa.blade.php]
	<!DOCTYPE html>
  <head>
      <title>Isi Data Diri</title>
      <link href="{{ asset("/bootstrap-3.3.7-dist/css/bootstrap.css") }}" rel="stylesheet" type="text/css" />
      <link href="{{ asset("/css/styles_list_surat.css") }}" rel="stylesheet" type="text/css">

  </head>

  <body>
    <div>
        <img id=banner src="{{ asset("/images/banner ftis.png") }}" />
    </div>


    <!-- Navigation here -->
    @include('mahasiswa.menu')

    <div class="container">
      <div class="main">
          <div class="row">
            <div class="col-md-8 content">
              <h1>Isi Data Diri Anda</h1>
              <br>
              <form action = "{{ url('/preview') }}" method="post" class="form-horizontal">
                <div class="form-group">
                  <label for="nama" class="col-sm-3">Nama</label>
                  <div class="col-sm-9">
                    <input type="text" class="form-control" id="nama" name="nama" value="{{ $user->nama_mahasiswa }}" readonly style="border: none" />
                  </div>
                </div>
                <div class="form-group">
                  <label for="prodi" class="col-sm-3">Program studi</label>
                  <div class="col-sm-9">
                    <span type="text" class="form-control" readonly style="border: none" >{{ $user->jurusan->nama_jurusan }}</span>
                    <input type="hidden" name="prodi" value="{{ $user->jurusan_id }}"/>
                  </div>
                </div>
                <div class="form-group">
                  <label for="npm" class="col-sm-3">NPM</label>
                  <div class="col-sm-9">
                    <input type="text" class="form-control" id="npm" name="npm" value="{{ $user->npm }}" readonly style="border: none">
                  </div>
                </div>
                <div class="form-group">
                  <label for="semester" class="col-sm-3">Semester</label>
                  <div class="col-sm-9">
                    <input type="text" class="form-control" name="semester" value="{{ $user->semester }}" readonly style="border: none" />
                  </div>
                </div>
                <div class="form-group">
                  <label for="thnAkademik" class="col-sm-3">Tahun akademik</label>
                  <div class="col-sm-9">
                    <input type="text" class="form-control" name="thnAkademik" value="{{ $user->thnAkademik }}" readonly style="border: none" />
                  </div>
                </div>
                <div class="form-group">
                  <label for="penyediabeasiswa" class="col-sm-3">Penyedia beasiswa</label>
                  <div class="col-sm-9">
                    <input type="text" class="form-control" id="penyediabeasiswa" required name="penyediabeasiswa" >
                  </div>
                </div>
                <input type="hidden" value="{{ $formatsurat_id }}" name="jenis_surat">
                {!! csrf_field() !!}
                <div class="form-group">
                  <div class="col-sm-offset-3 col-sm-10">
                    <button type="submit" class="btn btn-primary">Lanjutkan</button>
                  </div>
                </div>
              </form>
            </div>
            @include('mahasiswa.profile_bar')
          </div>
      </div>
    </div>
    <div class="footer">
    </div>
  </body>
</html>

\end{lstlisting}

\begin{lstlisting}[language=Java,basicstyle=\tiny,caption=Preview isi data keterangan beasiswa]
	<!DOCTYPE html>
  <head>
      <title>Preview</title>
      <link href="{{ asset("/bootstrap-3.3.7-dist/css/bootstrap.css") }}" rel="stylesheet" type="text/css" />
      <link href="{{ asset("/css/styles_list_surat.css") }}" rel="stylesheet" type="text/css">

  </head>

  <body>
    <div>
        <img id=banner src="{{ asset("/images/banner ftis.png") }}" />
    </div>


    <!-- Navigation here -->
    @include('mahasiswa.menu')

    <div class="container">
      <div class="main">
          <div class="row">
            <div class="col-md-8 contentPreview form-horizontal">
              <h4 style="font-weight:bold">SURAT PERNYATAAN</h4>
              <br>

                <form action = "{{ url('/kirimFormulir') }}" method="post">
                  <div class="form-group">
                    <label class="col-sm-3 prevLabel">Nama</label>
                    <div class="col-sm-9" name="nama">
                        {{ $nama }}
                    </div>
                  </div>
                  <div class="form-group">
                    <label class="col-sm-3 prevLabel">Program Studi</label>
                    <div class="col-sm-9" name="prodi">
                      <span>{{ $user->jurusan->nama_jurusan }}</span>
                      <input type="hidden" name="prodi" value="{{ $prodi }}"/>
                    </div>
                  </div>
                  <div class="form-group">
                    <label class="col-sm-3 prevLabel">NPM</label>
                    <div class="col-sm-9" name="npm">
                        {{ $npm }}
                    </div>
                  </div>
                  <div class="form-group">
                    <label class="col-sm-3 prevLabel">Semester</label>
                    <div class="col-sm-9" name="semester">
                        {{ $semester }}
                    </div>
                  </div>
                  <div class="form-group">
                    <label class="col-sm-3 prevLabel">Tahun Akademik</label>
                    <div class="col-sm-9" name="thnAkademik">
                        {{ $thnAkademik }}
                    </div>
                  </div>
                  <div class="form-group">
                    <label class="col-sm-3 prevLabel">Penyedia Beasiswa</label>
                    <div class="col-sm-9" name="penyediabeasiswa">
                        {{ $penyediabeasiswa }}
                    </div>
                  </div>
                  <input type="hidden" value="{{ $formatsurat_id }}" name="idFormat">
                  <input type="hidden" value="{{ $dataSurat }}" name="dataSurat">
                  <input type="hidden" value="{{ $penyediabeasiswa }}" name="provider">
                  {!! csrf_field() !!}
                  <br>
                  <div class="form-group">
                    <div class="col-sm-offset-3 col-sm-10">
                      <button class="btn btn-default" onclick="goBack()">Kembali</button>
                      <button type="submit" class="btn btn-success">Buat Surat</button>
                    </div>
                  </div>
                </form>
            </div>
            @include('mahasiswa.profile_bar')
          </div>
      </div>
    </div>
    <div class="footer">
    </div>
    <script>
      function goBack() {
          window.history.back();
      }
    </script>
  </body>
</html>

\end{lstlisting}

\begin{lstlisting}[language=Java,basicstyle=\tiny,caption=\textit{Navigation bar} untuk mahasiswa]
	<div class="navigation">
     <div class="navbar text-center">
        <ul class="inline">
           <a href="/home_mahasiswa"><li>Home</li></a>
           <a href="/pilih_kategori_surat"><li>Buat Surat</li></a>
           <a href="{{ url('/logout') }}"
               onclick="event.preventDefault();
                        document.getElementById('logout-form').submit();"><li>
               Logout
           <form id="logout-form" action="{{ url('/logout') }}" method="POST" style="display: none;">
               {{ csrf_field() }}
           </form>
           </li></a>
        </ul>
     </div>
</div>
<br/>
<!-- <div class="row">
  <div class="col-md-offset-1 col-sm-offset-1 col-md-4 col-sm-5">
    <div style="color:white">
        Selamat Datang, {!! Auth::user()->name !!}
    </div>
  </div>
</div> -->
<div class="row">
  <div class="col-md-offset-1 col-sm-offset-1 col-sm-4 col-md-4">

  </div>
</div>

\end{lstlisting}

\begin{lstlisting}[language=Java,basicstyle=\tiny,caption=\textit{Sidebar} untuk mahasiswa]
	<div class="col-md-4 profile">
  <div class="card hovercard">
      <div class="cardheader">

      </div>
      <div class="avatar">
          <img alt="" src="{{ $user->foto_mahasiswa }}" />
      </div>
      <div class="info">
          <div class="title">
              {{ $user->nama_mahasiswa }}
          </div>
          <div class="desc">{{ $user->npm }}</div>
          <div class="desc">{{ $user->jurusan->nama_jurusan }}</div>
      </div>
  </div>
</div>

\end{lstlisting}

\begin{lstlisting}[language=Java,basicstyle=\tiny,caption=\textit{Home} pejabat]
	<!DOCTYPE html>
  <head>
      <title>Home - Pejabat</title>
      <link href="{{ asset("/bootstrap-3.3.7-dist/css/bootstrap.css") }}" rel="stylesheet" type="text/css" />
      <link href="{{ asset("/css/styles_list_surat.css") }}" rel="stylesheet" type="text/css">

  </head>

  <body>
    <div>
        <img id=banner src="{{ asset("/images/banner ftis.png") }}" />
    </div>

    <!-- Navigation Here -->
    @include('pejabat.menu')

    <div class="container">
      <div class="main">
          <div class="row">
            <div class="col-md-8 content">
              <form>
                    <table>
                      <tr>
                        <td><label>Kata kunci :</label></td>
                        <td><label>Cari berdasarkan :</label></td>
                      </tr>

                      <tr>
                        <td class = "search">
                          <select name="kategori" class="form-control">
                            <option value="tanggalBuat">Cari semua surat</option>
                            <option value="noSurat">Nomor surat</option>
                            <option value="tanggalBuat">Tanggal pembuatan</option>
                            <option value="perihal">Perihal</option>
                            <option value="kepada]">Kepada</option>
                            <option value="nama">Pembuat surat</option>
                            <option value="idFormatSurat">Jenis surat</option>
                          </select>
                        </td>
                        <td class = "search">
                          <input type="text" name="searchBox" class="form-control" size="68" />
                        </td>
                        <td>
                          <input type="submit" name="findmail" class="btn btn-primary" value="Cari surat" />
                        </td>
                      </tr>
                    </table>
              </form>
              <br>
              <table class="table table-striped">
                <tr>
                  @if(count($pesanansurats) == 0)
                    <tr>
                        <td colspan="5" align="center">Tidak ada pesanan surat ...</td>
                    </tr>
                @else
                    <tr>
                      <th>JENIS SURAT</th>
                      <th>PERIHAL</th>
                      <th>PEMOHON</th>
                      <th>PENERIMA</th>
                      <th>TANGGAL PEMBUATAN</th>
                      <th>DATA SURAT</th>
                      <th>KONTROL</th>
                    </tr>
                    @foreach($pesanansurats as $pesanansurat)
                        <tr>
                          <td class="ctr">{{ $pesanansurat->formatsurat->jenis_surat }}</td>
                          <td class="ctr">{{ $pesanansurat->perihal }}</td>
                          <td class="ctr">{{ $pesanansurat->mahasiswa->nama_mahasiswa }}</td>
                          <td class="ctr">{{ $pesanansurat->penerimaSurat }}</td>
                          <td class="ctr">{{ $pesanansurat->created_at }}</td>
                          <td class="ctr"><textarea rows="5" cols="30" style="border: none" readonly>{{ $pesanansurat->dataSurat }}</textarea></td>
                          <td class="ctr">
                            @if($pesanansurat->formatsurat->id == 9 || $pesanansurat->formatsurat->id == 10)
                            <form action="/persetujuan" method="post">
                              <input type="hidden" value="{{ $pesanansurat->formatsurat_id }}" name="idFormatSurat">
                              <input type="hidden" value="{{ $pesanansurat->dataSurat }}" name="dataSurat">
                              <input type="hidden" value="{{ $pesanansurat->id }}" name="idPesananSurat">
                              {!! csrf_field() !!}
                              <button type="submit" class="btn btn-default">Tambah<br>Persetujuan</button>
                            </form>
                            @else
                              <!-- <span class="btn btn-default"></span> -->
                            @endif
                          </td>
                        </tr>
                    @endforeach
                  @endif
              </table>
            </div>
              @include('pejabat.profile_bar')
          </div>
      </div>
    </div>
    <div class="footer">
    </div>
  </body>
</html>

\end{lstlisting}

\begin{lstlisting}[language=php,basicstyle=\tiny,caption=Tambah persetujuan dan catatan]
	<!DOCTYPE html>
  <head>
      <title>Isi Catatan Dekan</title>
      <link href="{{ asset("/bootstrap-3.3.7-dist/css/bootstrap.css") }}" rel="stylesheet" type="text/css" />
      <link href="{{ asset("/css/styles_list_surat.css") }}" rel="stylesheet" type="text/css">

  </head>

  <body>
    <div>
        <img id=banner src="{{ asset("/images/banner ftis.png") }}" />
    </div>


    <!-- Navigation Here -->
    @include('pejabat.menu')

    <div class="container">
      <div class="main">
          <div class="row">
            <div class="col-md-8 content">
              <h1>Isi Persetujuan & Catatan</h1>
              <br>
              <form class="form-horizontal" method="post" action="/previewCatatan">
                <div class="form-group">
                  <label for="persetujuan" class="col-sm-3">Persetujuan</label>
                  <div class="col-sm-9">
                    <label class="radio-inline">
                      <input type="radio" name="persetujuan" value="Setuju" checked required>Setuju
                    </label>
                    <label class="radio-inline">
                      <input type="radio" name="persetujuan" value="Tidak setuju">Tidak setuju
                    </label>
                  </div>
                </div>
                <div class="form-group">
                  <label for="catatanDekan" class="col-sm-3">Catatan (Opsional)</label>
                  <div class="col-sm-9">
                    <textarea class="form-control" id="catatan" row="5" name="catatan"></textarea>
                  </div>
                </div>
                {!! csrf_field() !!}
                <input type="hidden" value="{{ $dataSurat }}" name="dataSurat">
                <input type="hidden" value="{{ $formatsurat_id }}" name="formatsurat_id">
                <input type="hidden" value="{{ $idPesananSurat }}" name="idPesananSurat">
                <div class="form-group">
                  <div class="col-sm-offset-3 col-sm-10">
                    <button type="submit" class="btn btn-primary">Lanjutkan</button>
                  </div>
                </div>
              </form>
            </div>
            @include('pejabat.profile_bar')
          </div>
      </div>
    </div>
    <div class="footer">
    </div>
  </body>
</html>

\end{lstlisting}

\begin{lstlisting}[language=php,basicstyle=\tiny,caption=\textit{Preview} isi persetujuan dan catatan]
	<!DOCTYPE html>
  <head>
      <title>Isi data diri</title>
      <link href="{{ asset("/bootstrap-3.3.7-dist/css/bootstrap.css") }}" rel="stylesheet" type="text/css" />
      <link href="{{ asset("/css/styles_list_surat.css") }}" rel="stylesheet" type="text/css">

  </head>

  <body>
    <div>
        <img id=banner src="{{ asset("/images/banner ftis.png") }}" />
    </div>


    <!-- Navigation here -->
    @include('mahasiswa.menu')

    <div class="container">
      <div class="main">
          <div class="row">
            <div class="col-md-8 contentPreview form-horizontal">
              <h4 style="font-weight:bold">FORMULIR PERMOHONAN CUTI STUDI</h4>
              <br>
              <form action = "{{ url('/updateFormulir') }}" method="post">
                <div class="form-group">
                  <label for="nama" class="col-sm-3 prevLabel">Nama</label>
                  <div class="col-sm-9" name="nama">
                    {{ $nama }}
                  </div>
                </div>
                <div class="form-group">
                  <label for="npm" class="col-sm-3 prevLabel">NPM</label>
                  <div class="col-sm-9" name="npm">
                    {{ $npm }}
                  </div>
                </div>
                <div class="form-group">
                  <label for="prodi" class="col-sm-3 prevLabel">Program Studi</label>
                  <div class="col-sm-9" name="prodi">
                    {{ $prodi }}
                  </div>
                </div>
                <div class="form-group">
                  <label for="fakultas" class="col-sm-3 prevLabel">Fakultas</label>
                  <div class="col-sm-9" name="fakultas">
                    {{ $fakultas }}
                  </div>
                </div>
                <div class="form-group">
                  <label for="alamat" class="col-sm-3 prevLabel">Alamat</label>
                  <div class="col-sm-9" name="alamat">
                    {{ $alamat }}
                  </div>
                </div>
                <div class="form-group prev">
                  <label for="alasanCutiStudi" class="col-sm-3 prevLabel">Alasan cuti studi ke </label>
                  <div class="col-sm-9" name="alasanCutiStudi">
                    {{ $cutiStudiKe }}<br>
                    {{ $alasanCutiStudi }}
                  </div>
                </div>
                <div class="form-group prev">
                  <label for="catatanDosenWali" class="col-sm-3 prevLabel">Catatan dosen wali </label>
                  <div class="col-sm-9" name="catatanDosenWali">
                    Nama : {{ $dosenWali }}<br>
                    {{ $persetujuanDosenWali }}<br>
                    {{ $catatanDosenWali }}
                    <input type="hidden" name="dosenWali" value="{{ $persetujuanDosenWali }}|{{ $catatanDosenWali }}" />
                  </div>
                </div>
                <div class="form-group prev">
                  <label for="catatanKaprodi" class="col-sm-3 prevLabel">Catatan Kaprodi </label>
                  <div class="col-sm-9" name="catatanKaprodi">
                    {{ $persetujuanKaprodi }}<br>
                    {{ $catatanKaprodi }}
                    <input type="hidden" name="kaprodi" value="{{ $persetujuanKaprodi }}|{{ $catatanKaprodi }}" />
                  </div>
                </div>
                <div class="form-group prev">
                  <label for="catatanWDII" class="col-sm-3 prevLabel">Catatan WD II</label>
                  <div class="col-sm-9" name="catatanWDII">
                    {{ $persetujuanWDII }}<br>
                    {{ $catatanWDII }}
                    <input type="hidden" name="wd2" value="{{ $persetujuanWDII }}|{{ $catatanWDII }}" />
                  </div>
                </div>
                <div class="form-group prev">
                  <label for="catatanWDI" class="col-sm-3 prevLabel">Catatan WD I</label>
                  <div class="col-sm-9" name="catatanWDI">
                    {{ $persetujuanWDI }}<br>
                    {{ $catatanWDI }}
                    <input type="hidden" name="wd1" value="{{ $persetujuanWDI }}|{{ $catatanWDI }}" />
                  </div>
                </div>
                <div class="form-group prev">
                  <label for="persetujuanDekan" class="col-sm-3 prevLabel">Persetujuan Dekan</label>
                  <div class="col-sm-9" name="persetujuanDekan" >
                    {{ $persetujuanDekan }}
                    <input type="hidden" name="dekan" value="{{ $persetujuanDekan }}" />
                  </div>
                </div>
                <div class="form-group prev">
                  <label for="semester" class="col-sm-3 prevLabel">Semester</label>
                  <div class="col-sm-9" name="semester">
                    {{ $semester }}
                  </div>
                </div>
                <div class="form-group prev">
                  <label for="thnAkademik" class="col-sm-3 prevLabel">Tahun Akademik</label>
                  <div class="col-sm-9" name="thnAkademik">
                    {{ $thnAkademik }}
                  </div>
                </div>
                <input type="hidden" value="{{ $formatsurat_id }}" name="idFormat">
                <input type="hidden" value="{{ $dataSurat }}" name="dataSurat">
                <input type="hidden" name="idPesanansurat" value="{{$idPesanansurat}}">

                {!! csrf_field() !!}
                <br>
                <div class="form-group">
                  <div class="col-sm-offset-3 col-sm-10">
                    <button class="btn btn-default" onclick="goBack()">Kembali</button>
                    <button type="submit" class="btn btn-success">Buat Surat</button>
                  </div>
                </div>
              </form>
            </div>
            @include('mahasiswa.profile_bar')
          </div>
      </div>
    </div>
    <div class="footer">
    </div>
    <script>
      function goBack() {
          window.history.back();
      }
    </script>
  </body>
</html>
	
\end{lstlisting}

\begin{lstlisting}[language=php,basicstyle=\tiny,caption=\textit{History} pejabat]
	<!DOCTYPE html>
  <head>
      <title>Data Mahasiswa</title>
      <link href="{{ asset("/bootstrap-3.3.7-dist/css/bootstrap.css") }}" rel="stylesheet" type="text/css" />
      <link href="{{ asset("/css/styles_list_surat.css") }}" rel="stylesheet" type="text/css">
  </head>

  <body>
    <div>
        <img id=banner src="{{ asset("/images/banner ftis.png") }}" />
    </div>

    <!-- Navigation Here -->
    @include('pejabat.menu')

    <div class="container">
      <div class="main">
          <div class="row">
            <div class="col-md-8 content">
              <form class="form-inline" action= "{{ url('/format_surat') }}" method="get">
                <div class="form-group">
                  <label for="kategori_format_surat">Cari berdasarkan :</label><br>
                  <select name="kategori_mahasiswa" class="form-control">
                    <option value="">Cari semua surat</option>
                    <option value="noSurat">Nomor Surat</option>
                    <option value="jenis_surat">Jenis Surat</option>
                    <option value="perihal">Perihal</option>
                    <option value="pemohon">Pemohon</option>
                    <option value="penerima">Penerima</option>
                    <option value="tanggalPembuatan">Tanggal Pembuatan</option>
                  </select>
                </div>
                <div class="form-group">
                  <label for="searchBox_format_surat">Kata kunci :</label><br>
                  <input type="text" name="searchBox" class="form-control" size="68" />
                  <button type="submit" name="findmail" class="btn btn-primary">Cari surat</button>
                </div>
              </form>
              <br>
              <table class="table table-striped table-hover">
                @if($historysurats != null)
                  @if(count($historysurats) == 0)
                      <tr>
                          <td colspan="5" align="center">Tidak ada history surat....</td>
                      </tr>
                  @else
                      <tr>
                        <th>NOMOR SURAT</th>
                        <th>JENIS SURAT</th>
                        <th>PERIHAL</th>
                        <th>PEMOHON</th>
                        <th>PENERIMA</th>
                        <th>TANGGAL PEMBUATAN</th>
                        <th>PENANDATANGANAN</th>
                        <th>PENGAMBILAN</th>
                      </tr>
                      @foreach($historysurats as $historysurat)
                        <tr>
                          <td class="ctr">{{ $historysurat->no_surat }}</td>
                          <td class="ctr">{{ $historysurat->formatsurat->jenis_surat }}</td>
                          <td class="ctr">{{ $historysurat->perihal }}</td>
                          <td class="ctr">{{ $historysurat->mahasiswa->nama_mahasiswa  }}</td>
                          <td class="ctr">{{ $historysurat->penerimaSurat }}</td>
                          <td class="ctr">{{ $historysurat->created_at }}</td>
                          <td align="center">
                            @if($historysurat->penandatanganan)
                              <button type="submit" disabled class="btn btn-success" style="display:block">Sudah</button>
                            @else
                              <form action="{{url('/ubahStatusPenandatanganan')}}" method="post">
                                <input type="hidden" value="{{ $historysurat->id }}" name="id">
                                {!! csrf_field() !!}
                                <button type="submit" class="btn btn-default " style="display:block">Belum</button>
                              </form>
                            @endif
                          </td>
                          <td class="ctr">
                            @if($historysurat->pengambilan)
                              <p>Sudah</p>
                            @else
                              <p>Belum</p>
                            @endif
                          </td>
                        </tr>
                      @endforeach
                  @endif
                @endif
              </table>
              <div style="text-align:center">{!! $historysurats->links() !!}</div>
            </div>
            @include('pejabat.profile_bar')
          </div>
      </div>
    </div>
    <div class="footer">
    </div>
  </body>
</html>

\end{lstlisting}

\begin{lstlisting}[language=php,basicstyle=\tiny,caption=\textit{Navigation bar} untuk pejabat]
	<div class="navigation">
     <div class="navbar text-center">
        <ul class="inline">
          <a href="/home_pejabat"><li>Home</li></a>
          <a href="/history_pejabat"><li>History Surat</li></a>

          <a href="{{ url('/logout') }}"
              onclick="event.preventDefault();
                       document.getElementById('logout-form').submit();"><li>
              Logout
          <form id="logout-form" action="{{ url('/logout') }}" method="POST" style="display: none;">
              {{ csrf_field() }}
          </form>
          </li></a>
        </ul>
     </div>
</div>
<br>
<!-- <div class="row">
  <div class="col-md-offset-1 col-sm-offset-1 col-md-4 col-sm-5">
    <div style="color:white">
        Selamat Datang, {!! Auth::user()->name !!}
    </div>
  </div>
</div> -->
<div class="row">
  <div class="col-md-offset-1 col-sm-offset-1 col-sm-4 col-md-4">

  </div>
</div>

\end{lstlisting}

\begin{lstlisting}[language=php,basicstyle=\tiny,caption=\textit{Sidebar} untuk pejabat]
	<div class="col-md-4 profile">
  <div class="card hovercard">
      <div class="cardheader">

      </div>
      <div class="avatar">
          <img alt="" src="http://simpleicon.com/wp-content/uploads/user1.png">
      </div>
      <div class="info">
          <div class="title">
              {{ $user->nama_dosen }}
          </div>
          <div class="desc">{{ $user->nik }}</div>
          <div class="desc">{{ $user->jurusan->nama_jurusan }}</div>
      </div>
  </div>
</div>

\end{lstlisting}

\begin{lstlisting}[language=php,basicstyle=\tiny,caption=\textit{Home} TU]
	<!DOCTYPE html>
  <head>
      <title>Home - TU</title>
      <link href="{{ asset("/bootstrap-3.3.7-dist/css/bootstrap.css") }}" rel="stylesheet" type="text/css" />
      <link href="{{ asset("/css/styles_list_surat.css") }}" rel="stylesheet" type="text/css">

  </head>

  <body>
    <div>
        <img id=banner src="{{ asset("/images/banner ftis.png") }}" />
    </div>

    @include('tu.menu')

    <div class="container">
      <div class="main">
          <div class="row">
            <div class="col-md-8 content">
              <form class="form-inline" action= "{{ url('/home_TU') }}" method="get">
                <div class="form-group">
                  <label for="kategori">Cari berdasarkan :</label><br>
                  <select name="kategori" class="form-control">
                    <option value="">Cari semua surat</option>
                    <option value="jenis_surat">Jenis Surat</option>
                    <option value="perihal">Perihal</option>
                    <option value="pemohonSurat">Pemohon Surat</option>
                    <option value="penerimaSurat">Penerima Surat</option>
                    <option value="tanggalBuat">Tanggal pembuatan</option>
                  </select>
                </div>
                <div class="form-group">
                  <label for="searchBox">Kata kunci :</label><br>
                  <input type="text" name="searchBox" class="form-control" size="69" />
                  <button type="submit" name="findmail" class="btn btn-primary">Cari surat</button>
                </div>
              </form>
              <br>
              <table class="table table-striped">
                @if(count($pesanansurats) == 0)
                    <tr>
                        <td colspan="5" align="center">Tidak ada pesanan surat ...</td>
                    </tr>
                @else
                    <tr>
                      <th>JENIS SURAT</th>
                      <th>PERIHAL</th>
                      <th>PEMOHON</th>
                      <th>PENERIMA</th>
                      <th>TANGGAL PEMBUATAN</th>
                      <th>DATA SURAT</th>
                      <th>KONTROL</th>
                    </tr>
                    @foreach($pesanansurats as $pesanansurat)
                        <tr>
                          <td class="ctr">{{ $pesanansurat->formatsurat->jenis_surat }}</td>
                          <td class="ctr">{{ $pesanansurat->perihal }}</td>
                          <td class="ctr">{{ $pesanansurat->mahasiswa->nama_mahasiswa }}</td>
                          <td class="ctr">{{ $pesanansurat->penerimaSurat }}</td>
                          <td class="ctr">{{ $pesanansurat->created_at }}</td>
                          <td class="ctr"><textarea rows="5" cols="30" style="border: none" readonly>{{ $pesanansurat->dataSurat }}</textarea></td>
                          <td class="ctr">
                            <form action="/proses_surat" method="post">
                              <input type="hidden" value="{{ $pesanansurat->id }}" name="id">
                              <input type="hidden" value="{{ $pesanansurat->formatsurat_id }}" name="idFormatSurat">
                              <input type="hidden" value="{{ $pesanansurat->dataSurat }}" name="prosesSurat">
                              {!! csrf_field() !!}
                              <button type="submit" class="btn btn-default">Tambah<br>Nomor<br>Surat</button>
                            </form>
                          </td>
                        </tr>
                    @endforeach
                  @endif
              </table>
              <div style="text-align:center">{!! $pesanansurats->links() !!}</div>
            </div>
          @include('tu.profile_bar')
          </div>
      </div>
    </div>
    <div class="footer">
        <div style="text-align:center">Copyright Dony Erlangga</div>
    </div>
  </body>
</html>

\end{lstlisting}

\begin{lstlisting}[language=php,basicstyle=\tiny,caption=Tambah nomor surat untuk surat keterangan beasiswa]
	<!DOCTYPE html>
  <head>
      <title>Tambah Nomor Surat & Generate PDF</title>
      <link href="{{ asset("/bootstrap-3.3.7-dist/css/bootstrap.css") }}" rel="stylesheet" type="text/css" />
      <link href="{{ asset("/css/styles_list_surat.css") }}" rel="stylesheet" type="text/css">

  </head>

  <body>
    <div>
        <img id=banner src="{{ asset("/images/banner ftis.png") }}" />
    </div>

    @include('tu.menu')

    <div class="container">
      <div class="main">
          <div class="row">
            <div class="col-md-8 content">
                <h3 style="font-weight:bold;">Preview Akhir dan Isi Nomor Surat</h3>
                <br>
                <form class="form-horizontal" action="{{ url('/generatePDF') }}" method="post">
                  <div class="form-group">
                    <label class="col-sm-3 prevLabel">Nama</label>
                    <div class="col-sm-9" name="nama">
                        {{ $nama }}
                    </div>
                  </div>
                  <div class="form-group">
                    <label class="col-sm-3 prevLabel">Program Studi</label>
                    <div class="col-sm-9" name="prodi">
                      <span>{{ $user->jurusan->nama_jurusan }}</span>
                      <input type="hidden" name="prodi" value="{{ $prodi }}"/>
                    </div>
                  </div>
                  <div class="form-group">
                    <label class="col-sm-3 prevLabel">NPM</label>
                    <div class="col-sm-9" name="npm">
                        {{ $npm }}
                    </div>
                  </div>
                  <div class="form-group">
                    <label class="col-sm-3 prevLabel">Semester</label>
                    <div class="col-sm-9" name="semester">
                        {{ $semester }}
                    </div>
                  </div>
                  <div class="form-group">
                    <label class="col-sm-3 prevLabel">Tahun Akademik</label>
                    <div class="col-sm-9" name="thnAkademik">
                        {{ $thnAkademik }}
                    </div>
                  </div>
                  <div class="form-group">
                    <label class="col-sm-3 prevLabel">Jenis Beasiswa</label>
                    <div class="col-sm-9" name="penyediabeasiswa">
                        {{ $penyediabeasiswa }}
                    </div>
                  </div>
                  <div class="form-group">
                      <label class="col-sm-3" for="noSurat">Nomor Surat</label>
                      <div class="col-sm-6">
                          <input type="text" class="form-control" name="noSurat" required />
                      </div>
                  </div>
                  <input type="hidden" value="{{ $dataSurat }}" id="format" name="data">
                  <input type="hidden" value="{{ $formatsurat_id }}" name="idFormatSurat">
                  <input type="hidden" value="{{ $pemesan }}" name="pemesan">
                  {!! csrf_field() !!}
                  <br>
                  <div class="form-group">
                    <div class="col-sm-offset-3 col-sm-10">
                      <button class="btn btn-default" onclick="goBack()">Kembali</button>
                      <button type="submit" class="btn btn-success">Buat Surat (PDF)</button>
                    </div>
                  </div>
                </form>
            </div>
            @include('tu.profile_bar')
          </div>
      </div>
    </div>
    <div class="footer">
         
    </div>
    <script>
      function goBack() {
          window.history.back();
      }

      function compile(){
        var pdftex = new PDFTeX();
        var latex_code =
      }
    </script>
  </body>
</html>
\end{lstlisting}

\begin{lstlisting}[language=php,basicstyle=\tiny,caption=\textit{History} TU]
	<!DOCTYPE html>
  <head>
      <title>History Surat - TU</title>
      <link href="{{ asset("/bootstrap-3.3.7-dist/css/bootstrap.css") }}" rel="stylesheet" type="text/css" />
      <link href="{{ asset("/css/styles_list_surat.css") }}" rel="stylesheet" type="text/css" />
  </head>

  <body>
    <div>
        <img id=banner src="{{ asset("/images/banner ftis.png") }}" />
    </div>

    @include('tu.menu')

    <div class="container">
      <div class="main">
          <div class="row">
            <div class="col-md-8 content">
              <form class="form-inline" action= "{{ url('/format_surat') }}" method="get">
                <div class="form-group">
                  <label for="kategori_format_surat">Cari berdasarkan :</label><br>
                  <select name="kategori_mahasiswa" class="form-control">
                    <option value="">Cari semua surat</option>
                    <option value="noSurat">Nomor Surat</option>
                    <option value="jenis_surat">Jenis Surat</option>
                    <option value="perihal">Perihal</option>
                    <option value="pemohon">Pemohon</option>
                    <option value="penerima">Penerima</option>
                    <option value="tanggalPembuatan">Tanggal Pembuatan</option>
                  </select>
                </div>
                <div class="form-group">
                  <label for="searchBox_format_surat">Kata kunci :</label><br>
                  <input type="text" name="searchBox" class="form-control" size="68" />
                  <button type="submit" name="findmail" class="btn btn-primary">Cari surat</button>
                </div>
              </form>
              <br>
              <table class="table table-striped table-hover">
                @if($historysurats != null)
                  @if(count($historysurats) == 0)
                      <tr>
                          <td colspan="5" align="center">Tidak ada history surat....</td>
                      </tr>
                  @else
                      <tr>
                        <th>NOMOR SURAT</th>
                        <th>JENIS SURAT</th>
                        <th>PERIHAL</th>
                        <th>PEMOHON</th>
                        <th>PENERIMA</th>
                        <th>TANGGAL PEMBUATAN</th>
                        <th>PENANDATANGANAN</th>
                        <th>PENGAMBILAN</th>
                      </tr>
                      @foreach($historysurats as $historysurat)
                        <tr>
                          <td class="ctr">{{ $historysurat->no_surat }}</td>
                          <td class="ctr">{{ $historysurat->formatsurat->jenis_surat }}</td>
                          <td class="ctr">{{ $historysurat->perihal }}</td>
                          <td class="ctr">{{ $historysurat->mahasiswa->nama_mahasiswa }}</td>
                          <td class="ctr">{{ $historysurat->penerimaSurat }}</td>
                          <td class="ctr">{{ $historysurat->created_at }}</td>
                          <td class="ctr">
                            @if($historysurat->penandatanganan)
                              <p>Sudah</p>
                            @else
                              <p>Belum</p>
                            @endif
                          </td>
                          <td align="center">
                            @if($historysurat->pengambilan)
                              <button type="submit" disabled class="btn btn-success">Sudah</button>
                            @else
                              @if($historysurat->penandatanganan == false)
                                <button type="submit" disabled class="btn btn-default">Belum</button>
                              @else
                                <form method="post" action="{{url('/ubahStatusPengambilan')}}">
                                  <input type="hidden" value="{{ $historysurat->id }}" name="id">
                                  {!! csrf_field() !!}
                                  <button type="submit" class="btn btn-default">Belum</button>
                                </form>
                              @endif
                            @endif
                          </td>
                        </tr>
                      @endforeach
                  @endif
                @endif
              </table>
              <div style="text-align:center">{!! $historysurats->links() !!}</div>
            </div>
              @include('tu.profile_bar')
          </div>
      </div>
    </div>
    <div class="footer">
         
    </div>
  </body>
</html>
\end{lstlisting}

\begin{lstlisting}[language=php,basicstyle=\tiny,caption=Data mahasiswa]
	<!DOCTYPE html>
  <head>
      <title>Data Mahasiswa</title>
      <link href="{{ asset("/bootstrap-3.3.7-dist/css/bootstrap.css") }}" rel="stylesheet" type="text/css" />
      <link href="{{ asset("/css/styles_list_surat.css") }}" rel="stylesheet" type="text/css">

  </head>

  <body>
    <div>
        <img id=banner src="{{ asset("/images/banner ftis.png") }}" />
    </div>

      @include('tu.menu')

    <div class="container">
      <div class="main">
          <div class="row">
            <div class="col-md-8 content">
              <!-- <a href="{{ URL::to('/tambah_data_mahasiswa') }}" class="btn btn-default">Tambah Data Mahasiswa</a> -->
              <form class="form-inline" action= "{{ url('/data_mahasiswa') }}" method="get">
                <div class="form-group">
                  <label for="kategori_mahasiswa">Cari berdasarkan :</label><br>
                  <select name="kategori_mahasiswa" class="form-control">
                    <option value="tanggalBuat">Cari semua surat</option>
                    <option value="nirm">NIRM</option>
                    <option value="npm">NPM</option>
                    <option value="nama_mahasiswa">Nama Mahasiswa</option>
                    <option value="prodi">Program Studi</option>
                    <option value="angkatan">Angkatan</option>
                    <option value="kota_lahir">Kota Lahir</option>
                    <option value="tanggal_lahir">Tanggal Lahir</option>
                  </select>
                </div>
                <div class="form-group">
                  <label for="searchBox">Kata kunci :</label><br>
                  <input type="text" name="searchBox" class="form-control" size = "65">
                  <button type="submit" name="findmail" class="btn btn-primary">Cari mahasiswa</button>
                </div>
              </form>
              <br>
              <table class="table table-striped table-hover">
                @if(count($mahasiswas) == 0)
                    <tr>
                        <td colspan="5" align="center">Tidak ada data mahasiswa ...</td>
                    </tr>
                @else
                    <tr>
                      <th>NIRM</th>
                      <th>NPM</th>
                      <th>NAMA MAHASISWA</th>
                      <th>PROGRAM STUDI</th>
                      <th>ANGKATAN</th>
                      <th>KOTA LAHIR</th>
                      <th>TANGGAL LAHIR</th>
                      <th>DOSEN WALI</th>
                      <th>FOTO</th>
                      <th>KONTROL</th>
                    </tr>
                    @foreach($mahasiswas as $mahasiswa)
                      <tr>
                        <td class="ctr">{{ $mahasiswa->nirm }}</td>
                        <td class="ctr">{{ $mahasiswa->npm }}</td>
                        <td class="ctr">{{ $mahasiswa->nama_mahasiswa }}</td>
                        <td class="ctr">{{ $mahasiswa->jurusan->nama_jurusan }}</td>
                        <td class="ctr">{{ $mahasiswa->angkatan }}</td>
                        <td class="ctr">{{ $mahasiswa->kota_lahir }}</td>
                        <td class="ctr">{{ $mahasiswa->tanggal_lahir }}</td>
                        <td class="ctr">{{ $mahasiswa->dosen->nama_dosen }}</td>
                        <td style="text-align:center"><a href = "{{ $mahasiswa->foto }}">klik disini</a></td>
                        <td>
                          <form action="/hapusMahasiswa" method="post">
                            <input type="hidden" value="{{ $mahasiswa->id }}" name="deleteID">
                            {!! csrf_field() !!}
                            <button type="submit" class="btn btn-danger" aria-label="Remove" data-toggle="tooltip" title="Remove">
                                <span class="glyphicon glyphicon-remove" aria-hidden="true"></span>
                            </button>
                          </form>
                        </td>
                      </tr>
                    @endforeach
                @endif
              </table>
            </div>
            @include('tu.profile_bar')
          </div>
      </div>
    </div>
    <div class="footer">
    </div>
  </body>
</html>

\end{lstlisting}

\begin{lstlisting}[language=php,basicstyle=\tiny,caption=Format surat]
	
<!DOCTYPE html>
  <head>
      <title>Format Surat</title>
      <link href="{{ asset("/bootstrap-3.3.7-dist/css/bootstrap.css") }}" rel="stylesheet" type="text/css" />
      <link href="{{ asset("/css/styles_list_surat.css") }}" rel="stylesheet" type="text/css" />
      <script src="{{ asset("/js/jquery-3.2.0.min.js") }}" type="text/javascript"></script>

  </head>

  <body>
    <div>
        <img id=banner src="{{ asset("/images/banner ftis.png") }}" />
    </div>

    @include('tu.menu')

    <div id="myModal" class="modal">
      <div class="modal-content" id="show">
        <span class="close">&times;</span>
      </div>
    </div>
    <div class="container">
      <div class="main">
          <div class="row">
            <div class="col-md-8 content" id ="test">
              <a href="{{ URL::to('/tambah_format_surat') }}" class="btn btn-default">Tambah Format Surat</a>
              <form class="form-inline" action= "{{ url('/format_surat') }}" method="get">
                <div class="form-group">
                  <label for="kategori_format_surat">Cari berdasarkan :</label><br>
                  <select name="kategori_format_surat" class="form-control">
                    <option value="">Cari semua surat</option>
                    <option value="idFormatSurat">ID Format Surat</option>
                    <option value="jenis_surat">Jenis Surat</option>
                    <option value="keterangan">Keterangan</option>
                  </select>
                </div>
                <div class="form-group">
                  <label for="searchBox_format_surat">Kata kunci :</label><br>
                  <input type="text" name="searchBox_format_surat" class="form-control" size = "65">
                  <button type="submit" name="findmail" class="btn btn-primary">Cari format surat</button>
                </div>
              </form>
              <br>
              <table class="table table-striped table-hover">
                @if($formatsurats != null)
                  @if(count($formatsurats) == 0)
                      <tr>
                          <td colspan="5" align="center">Tidak ada format surat ...</td>
                      </tr>
                  @else
                      <tr>
                        <th>ID FORMAT SURAT</th>
                        <th>JENIS SURAT</th>
                        <th>KETERANGAN</th>
                        <th>FILE SURAT</th>
                        <th colspan ="2"> KONTROL</th>
                      </tr>
                      @foreach($formatsurats as $formatsurat)
                        <tr>
                          <td class="ctr">{{ $formatsurat->idFormatSurat }}</td>
                          <td class="ctr">{{ $formatsurat->jenis_surat }}</td>
                          <td class="ctr">{{ $formatsurat->keterangan }}</td>
                          <td class="ctr">
                              <button type="submit" onclick="showModal({{ $formatsurat->id }})" class="btn btn-link">Klik disini</button>
                          </td>
                          <td class="ctr">
                            <form action="/hapusFormatsurat" method="post">
                              <input type="hidden" value="{{ $formatsurat->id }}" name="deleteID">
                              {!! csrf_field() !!}
                              <button type="submit" class="btn btn-danger" aria-label="Remove" data-toggle="tooltip" title="Remove">
                                  <span class="glyphicon glyphicon-remove" aria-hidden="true"></span>
                              </button>
                            </form>
                          </td>
                        </tr>
                      @endforeach
                  @endif
                @endif
              </table>
              <div style="text-align:center">{!! $formatsurats->links() !!}</div>
            </div>
            @include('tu.profile_bar')
          </div>
      </div>
    </div>
    <div class="footer">
    </div>
    <script type="text/javascript" src="{{ asset("js/jquery-3.2.0.min.js") }}"></script>
    <script>
        // Get the modal
        var modal = document.getElementById('myModal');
        var btn = document.getElementById("showFormatID");
        var span = document.getElementsByClassName("close")[0];

        function tutup() {
            modal.style.display = "none";
        }

        window.onclick = function(event) {
            if (event.target == modal) {
                modal.style.display = "none";
            }
        }
        function showModal(id){
          var jqxhr = $.get( "http://127.0.0.1:8000/Api/showFormatSurat?id=" + id, function() {
          })
          .done(function(data) {
            document.getElementById("show").innerHTML = `<span onclick='tutup()' class='close'>&times;</span>` + data.stringFormat;
            // $('#myModal').css({'position' : 'fixed'});
            modal.style.display = "block";
            modal.style.position = "fixed";
            // $('#test').css('position' : 'fixed');
          })
          .fail(function(error) {
            alert( "Error : " + error);
          });
        }
    </script>
  </body>
</html>

\end{lstlisting}

\begin{lstlisting}[language=php,basicstyle=\tiny,caption=Tambah format surat]
	
<!DOCTYPE html>
  <head>
      <title>Data Mahasiswa</title>
      <link href="{{ asset("/bootstrap-3.3.7-dist/css/bootstrap.css") }}" rel="stylesheet" type="text/css" />
      <link href="{{ asset("/css/styles_list_surat.css") }}" rel="stylesheet" type="text/css">

  </head>

  <body>
    <div>
        <img id=banner src="{{ asset("/images/banner ftis.png") }}" />
    </div>

      @include('tu.menu')

    <div class="container">
      <div class="main">
          <div class="row">
            <div class="col-md-8 content">
              <h1>Input Format Surat</h1>
              <br><br>
              <form class="form-horizontal" action="{{ url('/uploadFormat')}}" method="post"  enctype="multipart/form-data">
                <div class="form-group">
                  <label for="idFormatSurat" class="col-sm-3">ID Format Surat</label>
                  <div class="col-sm-9">
                    <input type="text" class="form-control" id="idFormatSurat" name="idFormatSurat" required>
                  </div>
                </div>
                <div class="form-group">
                  <label for="jenis_surat" class="col-sm-3">Jenis Surat</label>
                  <div class="col-sm-9">
                    <input type="text" class="form-control" id="jenis_surat" name="jenis_surat" required />
                  </div>
                </div>
                <div class="form-group">
                  <label class="col-sm-3" for="keterangan">Keterangan</label>
                  <div class="col-sm-9">
                    <textarea class="form-control" row="3" id="keterangan" name="keterangan" required></textarea>
                  </div>
                </div>
                <div class="form-group">
                  <label for="nama" class="col-sm-3">Upload Format Surat</label>
                  <div class="col-sm-9">
                    <input type="file" class="form-control" name="uploadFormat" required />
                  </div>
                </div>
                <div class="form-group">
                  <div class="col-sm-3"></div>
                  <div class="col-sm-9 checkbox">
                    <label><input type="checkbox" required/> Saya sudah yakin</label>
                  </div>
                </div>
                {!! csrf_field() !!}
                <div class="form-group">
                  <div class="col-sm-3"></div>
                  <div class="col-sm-9">
                    <button type="submit" class="btn btn-primary">Upload format surat (.TeX)</button>
                  </div>
                </div>
              </form>
            </div>
            @include('tu.profile_bar')
          </div>
      </div>
    </div>
    <div class="footer">
    </div>
  </body>
</html>

\end{lstlisting}

\begin{lstlisting}[language=php,basicstyle=\tiny,caption=\textit{Setting} semester dan thnAjaran.blade.php]
	<!DOCTYPE html>
  <head>
      <title>Setting</title>
      <link href="{{ asset("/bootstrap-3.3.7-dist/css/bootstrap.css") }}" rel="stylesheet" type="text/css" />
      <link href="{{ asset("/css/styles_list_surat.css") }}" rel="stylesheet" type="text/css">

  </head>

  <body>
    <div>
        <img id=banner src="{{ asset("/images/banner ftis.png") }}" />
    </div>

    @include('tu.menu')
      <div class="container">
        <div class="main">
            <div class="row">
              <div class="col-md-8 content">
                <h1>Atur Semester dan Tahun Ajaran</h1>
                <br>
                <form class="form-horizontal" method="post" action="/updateSemester">
                  <div class="form-group">
                    <label for="persetujuan" class="col-sm-3">Semester</label>
                    <div class="col-sm-9">
                      <label class="radio-inline">
                        <input type="radio" name="semester" value="Ganjil" checked required>Ganjil
                      </label>
                      <label class="radio-inline">
                        <input type="radio" name="semester" value="Genap">Genap
                      </label>
                    </div>
                  </div>
                  <div class="form-group">
                    <label for="catatanDekan" class="col-sm-3">Tahun Akademik</label>
                    <div class="col-sm-9">
                      <input type="text" class="form-control" name="thnAkademik">
                    </div>
                  </div>
                  {!! csrf_field() !!}
                  <div class="form-group">
                    <div class="col-sm-offset-3 col-sm-10">
                      <button type="submit" class="btn btn-primary">Lanjutkan</button>
                    </div>
                  </div>
                </form>
              </div>
              @include('tu.profile_bar')
            </div>
        </div>
      </div>
      <div class="footer">
      </div>
    </body>
  </html>

\end{lstlisting}

\begin{lstlisting}[language=php,basicstyle=\tiny,caption=\textit{Navigation bar} untuk petugas TU]
	<div class="navigation">
         <div class="navbar text-center">
            <ul class="inline">
               <a href="/home_TU"><li>Home</li></a>
               <a href="/history_TU"><li>History Surat</li></a>
               <a href="/data_mahasiswa"><li>Data Mahasiswa</li></a>
               <a href="/format_surat"><li>Format Surat</li></a>
               <a href="/setting"><li>Setting</li></a>
           <a href="{{ url('/logout') }}"
               onclick="event.preventDefault();
                        document.getElementById('logout-form').submit();"><li>
               Logout
           <form id="logout-form" action="{{ url('/logout') }}" method="POST" style="display: none;">
               {{ csrf_field() }}
           </form>
           </li></a>
        </ul>
     </div>
</div>
<br/>
<!-- <div class="row">
  <div class="col-md-offset-1 col-sm-offset-1 col-md-4 col-sm-5">
    <div style="color:white">
        Selamat Datang, {!! Auth::user()->name !!}
    </div>
  </div>
</div> -->
<div class="row">
  <div class="col-md-offset-1 col-sm-offset-1 col-sm-4 col-md-4">

  </div>
</div>

\end{lstlisting}

\begin{lstlisting}[language=php,basicstyle=\tiny,caption=\textit{Sidebar} untuk petugas TU]
	<div class="col-md-4 profile">
  <div class="card hovercard">
      <div class="cardheader">

      </div>
      <div class="avatar">
          <img alt="" src="http://simpleicon.com/wp-content/uploads/user1.png">
      </div>
      <div class="info">
          <div class="title">
              {{ $user->namaTU }}
          </div>
          <div class="desc">{{ $user->NIK }}</div>
      </div>
  </div>
</div>

\end{lstlisting}