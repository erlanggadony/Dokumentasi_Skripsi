\chapter{Analisis}
\label{chap:analisis}
Analisis membahas mengenai deskripsi sistem terkini, hasil survey aplikasi sejenis dan analisis sistem usulan.

\section{Deskripsi Sistem Terkini}
\label{sec:deskripsi_sistem_terkini}
Deskripsi sistem terkini membahas mengenai gambaran umum institusi, jenis-jenis surat akademik yang dikeluarkan oleh pihak TU, prosedur pemesanan surat akademik, prosedur pembuatan surat akademik, prosedur pengambilan surat akademik dan analisis kebutuhan. \\

\subsection{Gambaran Umum Institusi}
\label{sec:gambaran_umum_institusi}
Fakultas Teknologi Informasi dan Sains (FTIS) adalah salah satu fakultas yang terdapat di kampus Universitas Katolik Parahyangan (Unpar). FTIS dikepalai oleh seorang dekan yang dibantu oleh 3 wakil dekan. FTIS memiliki 3 program studi (prodi) yaitu prodi Matematika, Fisika dan Teknik Informatika. Setiap program studi dipimpin oleh seorang ketua prodi yang dibantu oleh seorang sekretaris prodi, kecuali jurusan fisika. Setiap prodi, kecuali prodi matematika, memiliki laboratorium yang dikepalai oleh seorang kepala laboratorium. Untuk mengurusi masalah administrasi fakultas, terdapat bagian Tata Usaha (TU) yang terdiri dari sub bagian (subag) tertentu yang mengurusi bidang tertentu.\

Gambar \hyperlink{organigram_fakultas}{3.1} menjelaskan struktur organisasi dari FTIS. Di bawah ini akan dijelaskan  setiap bagian dari struktur organisasi tersebut. Setiap fungsi pada struktur organisasi tersebut memiliki tugas, wewenang, dan tanggung jawab yang berbeda.
\begin{figure}[H]
	\centering
		\includegraphics[scale=0.65]{F:/Skripsi/Template/Gambar/Diagram/sistem_terkini/organigram/organigram_fakultas.JPG}
	\caption{Struktur Organisasi Fakultas Teknologi Informasi dan Sains}
	\label{fig:organigram_fakultas}
\end{figure}

\begin{enumerate}
	\item Dekan \\
	berfungsi sebagai pengambil keputusan tertinggi di fakultas.
	\item Senat Fakultas \\
	berfungsi sebagai partner dekan untuk membantu memberikan pertimbangan dari keputusan yang akan diambil. Senat dipilih dari dosen tetap di fakultas.
	\item Wakil Dekan Bidang Akademik (WD I)\\
	berfungsi untuk membantu dekan pada pengambilan segala keputusan yang berhubungan dengan kegiatan akademik yang berlangsung di setiap program studi yang ada pada fakultas.
	\item Wakil Dekan Bidang Sumber Daya (WD II)\\
	berfungsi untuk membantu dekan pada pengambilan segala keputusan yang berhubungan dengan pengelolaan setiap bentuk sumber daya yang terdapat di lingkungan fakultas.
	\item Wakil Dekan Bidang Kemahasiswaan dan Alumni (WD III)\\
	berfungsi untuk membantu dekan pada pengambilan segala keputusan yang berhubungan dengan setiap permasalahan kemahasiswaan dan alumni di lingkungan fakultas.
	\item Kepala Laboratorium \\
	bertanggung jawab akan perawatan dan operasional laboratorium.
	\item Ketua Program Studi \\
	berfungsi sebagai pengambil keputusan tertinggi di lingkungan program studi.
	\item Sekretaris Program Studi \\
	berfungsi untuk membantu ketua program studi pada pengambilan segala keputusan di lingkungan program studi.
	\item Tata Usaha \\
	berfungsi untuk melayani segala kegiatan administrasi yang terjadi di lingkungan FTIS.
	\item LPPM (Lembaga Penelitian \& Pengabdian kepada Masyarakat)\\
	berfungsi sebagai lembaga yang mengakomodasi penelitian dan pengabdian ilmu pengetahuan.
	\item Kepala Pusat Studi \\
	merupakan bagian kecil dari LPPM yang berfungsi untuk menerapkan Tridharma Perguruan Tinggi, namun lebih spesifik kepada suatu bidang studi.
\end{enumerate}

Gambar \hyperlink{organigram_TU}{3.2} merupakan struktur organisasi dari TU FTIS. Bagian tata usaha dipimpin oleh seorang Kepala Bagian (Kabag). Di bawah Kabag terdapat 4 sub bagian (Subag) yang dikepalai oleh seorang Kepala Sub Bagian (Kasubag). Berikut ini akan dijelaskan struktur organisasi yang ada di tata usaha FTIS.
\begin{figure}[H]
	\centering
		\includegraphics[scale=0.35]{F:/Skripsi/Template/Gambar/Diagram/sistem_terkini/organigram/organigram_TU.JPG}
	\caption{Struktur Organisasi Tata Usaha Fakultas Teknologi Informasi dan Sains}
	\label{fig:organigram_TU}
\end{figure}
\begin{enumerate}
	\item Kabag Tata Usaha \\
	berfungsi sebagai pemegang keputusan tertinggi di tata usaha.
	\item Kasubag Akademik \\
	berfungsi sebagai pengatur segala hal yang berhubungan dengan kegiatan akademik di lingkungan fakultas, seperti pengaturan jadwal kuliah dan ujian, jadwal perwalian, pembagian ruangan, memperbanyak dan mendistribusikan soal ujian, dll.
	\item Kasubag Umum dan Peralatan \\
	berfungsi sebagai pengatur segala kegiatan, sarana dan prasarana yang ada di lingkungan fakultas. 
	\item Kasubag Keuangan dan Kepegawaian \\
	berfungsi sebagai pengatur operasional keuangan dan kepegawaian di lingkungan fakultas.
	\item Kasubag Kemahasiswaan dan Alumni \\
	berfungsi untuk melayani segala keperluan kemahasiswaan dan alumni seperti keperluan surat menyurat, legalisir ijazah, mempersiapkan perlengkapan wisuda, dll.
\end{enumerate}
Berdasarkan uraian mengenai struktur organisasi fakultas, Kasubag Kemahasiswaan dan Alumni berperan mengurusi keperluan surat menyurat di fakultas. Surat-surat yang dibuat termasuk juga dengan surat akademik yang akan dibahas lebih detil pada subbab selanjutnya.

\subsection{Jenis-Jenis Surat Akademik}
\label{sec:jenis_jenis_surat_akademik}
Ada berbagai macam surat akademik yang dikeluarkan oleh TU FTIS. Surat yang dikeluarkan antara lain sebagai berikut :
\begin{enumerate}
	\item Surat perwalian yang diwakilkan. \\
	Surat ini berlaku apabila mahasiswa yang bersangkutan berhalangan hadir untuk melakukan perwalian dengan dosen wali sehingga diwakilkan kepada mahasiswa lain yang diberi kuasa.
	
	\item Surat izin cuti studi. \\
	Surat ini berlaku apabila mahasiswa yang bersangkutan tidak akan mengikuti perkuliahan pada semester tertentu. Biasanya surat ini digunakan apabila mahasiswa mengalami sakit keras dan membutuhkan masa penyembuhan yang lama ataupun tidak ada mata kuliah yang dibuka pada semester yang berjalan. Surat ini dapat diproses apabila mahasiswa pemohon telah memenuhi 2 syarat, yaitu tidak memiliki tunggakan pembayaran uang kuliah dan sudah membayar uang cuti studi.
	
	\item Surat dispensasi pembayaran. \\
	Surat ini berlaku apabila mahasiswa yang bersangkutan berencana mengambil mata kuliah kurang dari 10 sks pada semester yang akan ditempuh. Surat ini harus diproses sebelum masa FRS berlangsung. Surat ini berbentuk formulir isian. Pertama mahasiswa harus mengisi formulir. Setelah formulir diisi sesuai dengan data yang dibutuhkan, formulir akan diteruskan kepada dosen wali dari mahasiswa yang bersangkutan melalui Petugas TU untuk mendapatkan tanda tangan dosen wali. Setelah mendapat tanda tangan dosen wali, formulir diteruskan kepada Biro Keuangan Unpar untuk mendapatkan persetujuan. Apabila disetujui, akan dikirimkan surat persetujuan kepada Petugas TU untuk mengubah jumlah tagihan pembayaran uang kuliah milik mahasiswa di situs \textit{studentportal}. Sehingga jumlah tagihan yang muncul akan sesuai dengan sks mata kuliah yang akan diambil. 
	
	\item Surat pengajuan pengambilan kelebihan pembayaran uang kuliah (tunai). \\
	Surat ini berlaku apabila mahasiswa yang bersangkutan telah membayar uang kuliah sesuai dengan sks mata kuliah yang diambil namun kemudian membatalkan mata kuliah yang telah diambil tersebut. Surat ini berbentuk formulir isian. Pertama mahasiswa harus mengisi formulir. Setelah formulir diisi sesuai dengan data yang dibutuhkan, formulir akan ditandatangani oleh Petugas TU. Setelah ditandatangani, formulir diteruskan kepada Biro Keuangan Unpar untuk diproses. Setelah diproses akan dikirimkan surat keterangan pengembalian uang kuliah beserta sejumlah uang kelebihan pembayaran kuliah kepada Petugas TU. Uang akan diberikan kepada mahasiswa pemohon dan surat keterangan akan disimpan di TU sebagai arsip.
	
	\item Surat izin pengunduran diri mahasiswa. \\
	Surat ini berfungsi sebagai surat pernyataan apabila seorang mahasiswa hendak mengundurkan diri dari Unpar. Biasanya lasan pengambilan surat ini dikarenakan mahasiswa yang bersangkutan merasa tidak cocok dengan jurusan atau telah diterima di universitas lain dan hendak berkuliah di universitas tersebut.
	
	\item Surat keterangan. \\
	Surat ini memiliki fungsi yang beragam. Surat ini dapat digunakan untuk pernyataan mahasiswa aktif, pembuatan surat kelakuan baik, membuka rekening, membuat visa, dll.
	
	\item Surat penundaan pembayaran uang kuliah. \\
	Surat ini berlaku apabila mahasiswa yang bersangkutan belum bisa membayar uang kuliah sampai tanggal yang telah ditentukan. Fungsi surat ini yaitu untuk memberikan kelonggaran waktu pembayaran uang kuliah pada mahasiswa sehingga mahasiswa yang bersangkutan akan terlepas dari denda keterlambatan pembayaran. Pertama mahasiswa harus mengisi formulir. Setelah formulir diisi sesuai dengan data yang dibutuhkan, akan dibuatkan surat yang kemudian diteruskan kepada Wakil Dekan Bidang Sumber Daya melalui Petugas TU untuk mendapatkan tanda tangan wakil dekan. Setelah mendapat tanda tangan wakil dekan, surat diteruskan kepada Biro Keuangan Unpar untuk mendapatkan persetujuan. Apabila disetujui, akan dikirimkan surat persetujuan kepada Petugas TU yang menyatakan pemberian izin kepada mahasiswa pemohon untuk menunda pembayaran uang kuliah.
	
	\item Surat izin studi lapangan. \\
	Surat ini berfungsi sebagai surat pengantar apabila ada mahasiswa yang hendak melakukan wawancara, survei, studi banding atau observasi ke sebuah instansi.
	
	\item Surat permohonan beasiswa. \\
	Surat ini berfungsi apabila seorang mahasiswa hendak mengajukan beasiswa kepada universitas maupun fakultas. Surat ini berbentuk formulir isian. Setelah formulir diisi sesuai dengan data yang dibutuhkan, formulir diteruskan kepada dosen wali mahasiswa pemohon untuk mendapat catatan dan tanda tangan dosen wali. Setelah itu surat akan diteruskan kepada dekan untuk catatan dan tanda tangan dekan. Setelah semua data terisi lengkap, formulir akan diserahkan kepada Badan Kemahasiswaan dan Alumni Unpar.
	
	\item Surat keterangan beasiswa. \\
	Surat ini berfungsi sebagai surat pernyataan bahwa mahasiswa yang bersangkutan telah menerima beasiswa dari pihak non-Unpar dan tidak menerima beasiswa lain yang berasal dari Unpar.
\end{enumerate}

\subsection{Prosedur Pemesanan Surat}
\label{sec:prosedur_pemesanan_surat}
Gambar \hyperlink{pemesanan_terkini}{3.3} merupakan prosedur pemesanan surat yang dilakukan oleh mahasiswa kepada Kasubag Kemahasiswaan dan Alumni. Untuk selanjutnya Kasubag Kemahasiswaan dan Alumni akan disebut sebagai Petugas TU untuk mempersingkat penyebutan nama. Prosedur pemesanan surat dimulai dengan:
\begin{enumerate}
	\item Mahasiswa mendatangi Petugas TU dan menyebutkan surat yang dibutuhkan.
	\item Petugas TU memberikan formulir yang harus diisi sesuai dengan surat yang dibutuhkan.
	\item Mahasiswa mengisi formulir.
	\item Mahasiswa mengembalikan formulir ke Petugas TU.
	\item Petugas TU mengecek pengisian formulir. Apabila ada kesalahan pengisian, maka formulir dikembalikan kepada mahasiswa untuk diperbaiki. Jika tidak ada kesalahan maka dapat lanjut ke proses berikutnya.
	\item Apabila sedang tidak ada pesanan surat, surat dapat langsung dibuatkan, apabila ada pesanan, surat akan masuk \textit{waiting list} dan Petugas TU akan memberikan estimasi waktu selesai pengerjaan surat.
\end{enumerate}

\begin{figure}[H]
	\centering
		\includegraphics[scale=0.25]{F:/Skripsi/Template/Gambar/Diagram/sistem_terkini/work_flow/pemesanan_terkini.jpg}
	{\caption{Prosedur pemesanan surat terkini}}
	\label{fig:pemesanan_terkini}
\end{figure}

\subsection{Prosedur Pembuatan Surat}
\label{sec:prosedur_pembuatan_surat}
Gambar \hyperlink{pembuatan_terkini}{3.4} merupakan prosedur pembuatan surat yang dilakukan oleh Petugas TU. Prosedur pembuatan surat dimulai dengan:
\begin{enumerate}
	\item Petugas TU mengecek formulir yang telah diisi oleh mahasiswa.
	\item Petugas TU membuka \textit{template} surat yang dipesan.
	\item Petugas TU memasukkan semua data yang telah dituliskan oleh mahasiswa pada formulir ke \textit{template}.
	\item Petugas TU melakukan cek ulang apabila terjadi kesalahan dalam memasukkan data.
	\item Apabila tidak ada kesalahan Petugas TU dapat langsung mencetak surat.
	\item Petugas akan menghubungi pejabat tertentu yang memiliki keterkaitan dengan surat yang sedang diproses untuk mendapatkan tanda tangan dari pejabat yang bersangkutan.
	\item Apabila mahasiswa pemohon menunggu di sekitar ruang TU maka surat dapat langsung diambil oleh mahasiswa pemohon. Apabila tidak ada, maka surat akan disimpan oleh Petugas TU.
\end{enumerate}

\begin{figure}[H]
	\centering
		\includegraphics[scale=0.25]{F:/Skripsi/Template/Gambar/Diagram/sistem_terkini/work_flow/pembuatan_terkini.jpg}
	{\caption{Prosedur pembuatan surat terkini}}
	\label{fig:pembuatan_terkini}
\end{figure}

\subsection{Prosedur Pengambilan Surat}
\label{sec:prosedur_pengambilan_surat}
Gambar \hyperlink{pengambilan_terkini}{3.5} merupakan prosedur pengambilan surat yang dilakukan oleh mahasiswa kepada Petugas TU. Prosedur pengambilan surat dimulai dengan:
\begin{enumerate}
	\item Mahasiswa yang memesan mendatangi Petugas TU dan menyebutkan surat apa yang telah dipesan.
	\item Petugas TU memberikan surat yang telah dicetak kepada mahasiswa.
\end{enumerate}

\begin{figure}[H]
	\centering
		\includegraphics[scale=0.35]{F:/Skripsi/Template/Gambar/Diagram/sistem_terkini/work_flow/pengambilan_terkini.jpg}
	{\caption{Prosedur pengambilan surat terkini}}
	\label{fig:pengambilan_terkini}
\end{figure}

\subsection{Analisis Kebutuhan}
\label{sec:analisis_kebutuhan}
Berdasarkan paparan yang telah disebutkan di atas, maka ditemukan beberapa masalah yang menjadi fokus utama dari penelitian ini. Masalah-masalah yang telah ditemukan antara lain:
\begin{enumerate}
	\item Pemesanan surat akademik tidak bisa dilakukan mendadak.
	\item Petugas bisa kewalahan apabila permintaan surat yang masuk cukup banyak dalam sehari.
	\item Ada kemungkinan kesalahan pemasukan data dari formulir ke komputer.
\end{enumerate}

\section{Hasil Survey Aplikasi Sejenis}
\label{sec:hasil_survey_aplikasi_sejenis}
Berdasarkan survey yang telah dilakukan terhadap beberapa universitas di Bandung, seperti Institut Teknologi Bandung (ITB) dan Universitas Pendidikan Indonesia (UPI), tidak ditemukan adanya penggunaan aplikasi yang khusus untuk menangani kebutuhan surat akademik. Penanganan surat akademik masih dilakukan secara konvensional yaitu dengan mendatangi dosen wali terlebih dahulu dan apabila sudah mendapat persetujuan dari dosen wali maka mahasiswa dapat membuat surat akademik dibantu oleh tata usaha di fakultas masing-masing.

\section{Analisis Sistem Usulan}
\label{sec:analisis_sistem_usulan}
Analisis sistem usulan membahas mengenai prosedur pembuatan surat usulan, diagram \textit{usecase}, \textit{task scenario}, analisis kebutuhan data dari surat, dan diagram ER. \\

\subsection{Surat-surat Yang Dapat Ditangani}
\label{sec:surat_yang_ditangani}
Dari 10 surat yang telah disebutkan di atas, diambil 7 surat saja yang akan dibahas lebih lanjut pada penelitian ini. Hal ini dikarenakan 6 surat tersebut dapat menghasilkan surat yang kemudian akan dikembalikan kepada mahasiswa yang bersangkutan untuk kemudian disampaikan kepada lembaga yang membutuhkan surat tersebut. Keenam surat tersebut yaitu :
\begin{enumerate}
	\item Surat keterangan beasiswa. \\
	Surat pernyataan bahwa mahasiswa ybs. tidak menerima beasiswa dari pihak universitas maupun fakultas dan hanya menerimah beasiswa dari satu organisasi.
	\item Surat keterangan mahasiswa aktif. \\
	Surat pernyataan bahwa mahasiswa ybs. merupakan mahasiswa aktif Fakultas Teknologi Informasi dan Sains Uiversitas Katolik Parahyangan.
	\item Surat pengantar pembuatan visa. \\
	Surat yang pengantar yang diajukan mahasiswa kepada suatu kedutaan besar negara asing dengan tujuan untuk untuk membuat visa.
	\item Surat izin studi lapangan. \\
	Surat izin yang diajukan apabila seorang mahasiswa hendak melakukan penelitian ke suatu organisasi dan membutuhkan surat pengantar dari fakultas sebagai syarat penelitian. Surat ini dapat digunakan untuk perorangan maupun kelompok. Untuk surat izin studi lapangan kelompok, dapat mewakili 1 orang ketua dan maksimal 4 orang anggota.
	\item Surat perwalian yang diwakilkan. \\
	Surat yang diajukan oleh mahasiswa apabila mahasiswa ybs. berhalangan hadir pada saat perwalian dan hendak diwakilkan oleh mahasiswa lain. 
	\item Surat izin cuti studi. \\
	Surat yang diajukan oleh mahasiswa apabila mahasiswa ybs. hendak mengambil cuti studi pada semester tertentu. Surat ini memerlukan lampiran dari orang tua/wali untuk
	\item Surat izin pengunduran diri mahasiswa. \\
	Surat yang diajukan oleh mahasiswa apabila mahasiswa ybs. hendak mengundurkan diri dari Fakultas Teknologi Informasi dan Sains Uiversitas Katolik Parahyangan.
\end{enumerate}

\subsection{Prosedur Pembuatan Surat Usulan}
\label{sec:pembuatan_surat_usulan}
Gambar \hyperlink{pembuatan_usulan}{3.6} merupakan prosedur pembuatan surat yang dilakukan oleh mahasiswa. Pada prosedur usulan ini, untuk membuat surat akademik tidak perlu melalui 3 tahapan seperti yang dijelaskan pada subbagian sistem terkini. Prosedur pembuatan surat dimulai dengan:
\begin{enumerate}
	\item Mahasiswa melakukan login
	\item Mahasiswa memilih jenis surat yang akan dibuat
	\item Mahasiswa memasukkan data yang dibutuhkan dengan benar
	\item Mahasiswa mengklik tombol \textit{preview} untuk melihat surat secara keseluruhan
	\item Apabila terdapat kesalahan pada pengisian data, mahasiswa dapat mengklik tombol \textit{edit} untuk memperbaiki kesalahan pada pengisian data.
	\item Apabila tidak ada kesalahan, mahasiswa dapat mengklik tombol \textit{process} untuk meng-\textit{generate} \textit{file} .pdf dan men-\textit{download} \textit{file} tersebut.
	\item Setelah mendapatkan \textit{file} .pdf surat, mahasiswa dapat mencetak surat.
	\item Setelah surat dicetak, mahasiswa memberikan surat tersebut kepada Petugas TU untuk ditanda tangani oleh pejabat yang berkaitan dengan surat.
	\item Setelah surat ditanda tangani, Petugas TU mengembalikan surat yang sudah ditanda tangani tersebut kepada mahasiswa.
	\
\end{enumerate}
\begin{figure}[H]
	\centering
		\includegraphics[scale = 0.25]{F:/Skripsi/Template/Gambar/Diagram/sistem_usulan/flowchart/Flowchart0.jpg}
	{\caption{Prosedur pembuatan surat usulan}}
	\label{fig:pembuatan_usulan}
\end{figure}

\subsection{\textit{Data Flow Diagram (DFD)}}
\label{sec:data_flow_diagram}
\textit{Data Flow Diagram (DFD)} adalah diagram untuk memodelkan setiap aliran data pada sistem yang akan dibangun.
Gambar \hyperlink{data_flow}{3.7} merupakan diagram \textit{context} atau DFD \textit{level} 0 yang menjelaskan 
keseluruhan sistem beserta aktornya.

\begin{figure}[H]
	\centering
		\includegraphics[scale = 0.5]{F:/Skripsi/Template/Gambar/Diagram/sistem_usulan/dfd/Lv0.png}
	\caption{Diagram \textit{context}}
	\label{fig:data_flow}
\end{figure}

\textit{Website} penyedia surat akademik ini memiliki 3 aktor, yaitu \textit{mahasiswa}, \textit{pejabat} dan \textit{petugas TU}. Tiap aktor memiliki hak akses ke beberapa fitur tertentu. Gambar \hyperlink{level_1}{3.8} merupakan DFD \textit{level} 1 yang menjelaskan aliran data pada tiap aktor.

\begin{figure}[H]
	\centering
		\includegraphics[scale = 0.35]{F:/Skripsi/Template/Gambar/Diagram/sistem_usulan/dfd/Lv1.png}
	\caption{DFD \textit{level} 1}
	\label{fig:level_1}
\end{figure}

Gambar \hyperlink{level_2-1}{3.9} merupakan DFD \textit{level} 2-1 yang menjelaskan aliran data pada aktor mahasiswa.

\begin{figure}[H]
	\centering
		\includegraphics[scale = 0.25]{F:/Skripsi/Template/Gambar/Diagram/sistem_usulan/dfd/Level_2-1_Mahasiswa.jpg}
	\caption{DFD \textit{level} 2-1 untuk mahasiswa}
	\label{fig:level_2-1}
\end{figure}

Gambar \hyperlink{level_2-2}{3.10} merupakan DFD \textit{level} 2-2 yang menjelaskan aliran data pada aktor pejabat.

\begin{figure}[H]
	\centering
		\includegraphics[scale = 0.25]{F:/Skripsi/Template/Gambar/Diagram/sistem_usulan/dfd/Level_2-2_Pejabat.jpg}
	\caption{DFD \textit{level} 2-2 untuk pejabat}
	\label{fig:level_2-2}
\end{figure}

Gambar \hyperlink{level_2-3(1)}{3.11} dan \hyperlink{level_2-3(2)}{3.12} merupakan DFD \textit{level} 2-3 yang menjelaskan aliran data pada aktor petugas TU.

\begin{figure}[H]
	\centering
		\includegraphics[scale = 0.25]{F:/Skripsi/Template/Gambar/Diagram/sistem_usulan/dfd/Level_2-3_Petugas_TU_(1).jpg}
	\caption{DFD \textit{level} 2-3 untuk petugas TU (1)}
	\label{fig:level_2-3(1)}
\end{figure}

\begin{figure}[H]
	\centering
		\includegraphics[scale = 0.25]{F:/Skripsi/Template/Gambar/Diagram/sistem_usulan/dfd/Level_2-3_Petugas_TU_(2).jpg}
	\caption{DFD \textit{level} 2-3 untuk petugas TU (2)}
	\label{fig:level_2-3(2)}
\end{figure}

Gambar \hyperlink{level_3-1}{3.12} merupakan DFD \textit{level} 3-1 yang menjelaskan aliran data pada aktor petugas TU pada saat mengelola format surat akademik.

\begin{figure}[H]
	\centering
		\includegraphics[scale = 0.25]{F:/Skripsi/Template/Gambar/Diagram/sistem_usulan/dfd/Level_3-1_Kelola_Format_Surat.jpg}
	\caption{DFD \textit{level} 3-1 untuk mengelola surat akademik}
	\label{fig:level_3-1}
\end{figure}

Gambar \hyperlink{level_3-2}{3.13} merupakan DFD \textit{level} 3-2 yang menjelaskan aliran data pada aktor petugas TU pada saat mengelola data mahasiswa.

\begin{figure}[H]
	\centering
		\includegraphics[scale = 0.25]{F:/Skripsi/Template/Gambar/Diagram/sistem_usulan/dfd/Level_3-2_Kelola_Data_Mahasiswa.jpg}
	\caption{DFD \textit{level} 3-2 untuk mengelola data mahasiswa}
	\label{fig:level_3-2}
\end{figure}

\subsection{Analisis Kebutuhan Data}
\label{sec:analisis_kebutuhan_data}
Setiap surat memiliki kebutuhan data yang berbeda-beda bergantung dari jenis surat yang akan dibuat. Tabel 3.16-3.20 berikut ini akan menjelaskan data apa saja yang perlu diisikan oleh mahasiswa dalam proses pembuatan surat akademik.\
\begin{table}[H]
\centering
\caption{Tabel Kebutuhan Data Surat Perwakilan Perwalian}
\label{surat_perwakilan_perwalian}
\begin{tabular}{|l|}
\hline
\textbf{Atribut Data}                     \\ \hline
Semester            											\\ \hline
Tahun akademik								            \\ \hline
Nama mahasiswa yang diwakilkan            \\ \hline 
NPM mahasiswa yang diwakilkan             \\ \hline 
Program Studi mahasiswa yang diwakilkan   \\ \hline 
Nama mahasiswa yang diberi kuasa          \\ \hline 
NPM mahasiswa yang diberi kuasa           \\ \hline 
Program studi mahasiswa yang diberi kuasa \\ \hline 
Alasan tidak hadir perwalian               \\ \hline 
Mata kuliah yang diambil                  \\ \hline 
Tanggal pembuatan surat                   \\ \hline
Nama dosen wali								            \\ \hline
Nama wakil dekan I								        \\ \hline
\end{tabular}
\end{table}
\
\begin{table}[H]
\centering
\caption{Tabel Kebutuhan Data Surat Izin Pengunduran Diri}
\label{surat_izin_pengunduran_diri}
\begin{tabular}{|l|}
\hline
{\textbf{Atribut Data}}                     \\ \hline
{Nama mahasiswa}                            \\ \hline 
{NPM}                                       \\ \hline 
{NIRM}                                      \\ \hline 
{Alamat (tetap/di Badung)}                  \\ \hline 
{Nomor telepon (HP)}                        \\ \hline 
{Nama orang tua}                            \\ \hline 
{Tanggal surat}                             \\ \hline 
{Semester berhenti}                         \\ \hline 
{Catatan Dosen Wali}                        \\ \hline 
{Catatan Ketua Program Studi}               \\ \hline 
{Catatan Wakil Dekan I}                     \\ \hline 
{Catatan Wakil Dekan II}                    \\ \hline 
{Catatan Dekan}                             \\ \hline
\end{tabular}
\end{table}
\
\begin{table}[H]
\centering
\caption{Tabel Kebutuhan Data Surat Keterangan}
\label{surat_keterangan}
\begin{tabular}{|l|}
\hline
{\textbf{Atribut Data}}                     \\ \hline
{Nama mahasiswa}                            \\ \hline 
{NPM}                                       \\ \hline 
{Program studi}                             \\ \hline 
{Tempat, tanggal lahir}                     \\ \hline 
{Alamat di Bandung}                         \\ \hline 
{E-mail (selain e-mail Unpar)}               \\ \hline 
{Jenis surat yang akan dibuat}              \\ \hline 
{Alasan pembuatan surat}                    \\ \hline 
{Tahun akademik}                            \\ \hline 
{Nomor telepon (HP)}                        \\ \hline

\end{tabular}
\end{table}
\
\begin{table}[H]
\centering
\caption{Tabel Kebutuhan Data Surat Keterangan Beasiswa}
\label{surat_keterangan_beasiswa}
\begin{tabular}{|l|}
\hline
{\textbf{Atribut Data}}                     \\ \hline
{Nama mahasiswa}                            \\ \hline 
{NPM}                                       \\ \hline 
{Program studi}                             \\ \hline
{Nomor HP}			                            \\ \hline
{Alamat}        			                     	\\ \hline
{Jenis beasiswa}                            \\ \hline 
{Tanggal}                                   \\ \hline 
{Nama orang tua}                            \\ \hline
\end{tabular}
\end{table}
\
\begin{table}[H]
\centering
\caption{Tabel Kebutuhan Data Surat Izin Cuti Studi}
\label{surat_izin_cuti_studi}
\begin{tabular}{|l|}
\hline
{\textbf{Atribut Data}}                     \\ \hline
{Nama mahasiswa}                            \\ \hline 
{NPM}                                       \\ \hline 
{Fakultas}                                  \\ \hline 
{Program studi}                             \\ \hline 
{Alamat tetap}                              \\ \hline
{Cuti studi ke-n}   			                	\\ \hline 
{Alasan cuti studi ke-n}                    \\ \hline
{Dosen Wali}				                        \\ \hline 
{Catatan Dosen Wali}                        \\ \hline
{Catatan Ketua Progam Studi}                \\ \hline 
{Rekomendasi Pembantu Dekan II}             \\ \hline 
{Rekomendasi Pembantu Dekan I}              \\ \hline 
{Semester}                          		    \\ \hline 
{Tahun akademik}                            \\ \hline
{Persetujuan Dekan}                         \\ \hline 
{Tanggal surat}                             \\ \hline
\end{tabular}
\end{table}
