\chapter{Implementasi dan Pengujian}
\label{chap:implementasi_dan_pengujian}

\section{Lingkungan Implementasi}
\label{sec:lingkungan_implementasi}

\subsection{Lingkungan Perangkat Keras}
\label{sec:lingkungan_perangkat_keras}
Untuk membangun website penyedia surat akademik, spesifikasi perangkat keras yang digunakan sebagai berikut :
\begin{itemize}
	\item Processor : Intel(R) Core(TM)i7-4702MQ CPU @2.20GHz
	\item Memory : DDR3 8 GB
	\item Harddisk : 1 TB
	\item VGA :
	\item Monitor
	\item Keyboard dan Mouse
\end{itemize}

\subsection{Lingkungan Perangkat Lunak}
\label{sec:lingkungan_perangkat_lunak}
Untuk membangun website penyedia surat akademik, spesifikasi perangkat lunak yang digunakan sebagai berikut :
\begin{itemize}
	\item Web server : Apache
	\item Tools : XAMPP Control Panel v3.2.2
	\item Bahasa pemrograman : PHP, Javascript
	\item Framework : Laravel 5.3
	\item Database management system : MySQL 
	\item Operation system : Windows 10 Education 64-bit
\end{itemize}

\section{Pengujian}
\label{sec:pengujian}
Bagian ini membahas mengenai pengujian yang dilakukan dari website penyedia surat akademik yang dibangun. Metode pengujian yang digunakan yaitu metode \textit{black box testing}, yaitu metode pengujian \textit{software} dengan mencoba sebanyak mungkin contoh kasus ke dalam sistem tanpa melihat kode program untuk menemukan kesalahan. Status dari pengujian ini terbagi 2, yaitu OK dan GAGAL.

\begin{itemize}
	\item Pengujian \textit{login}
	\begin{table}[H]
	\centering
	\caption{Pengujian \textit{Login}}
	\label{pengujian_login}
	\begin{tabular}{|l|l|l|l|l|l|}
	\hline
	\textbf{No.}&\textbf{Langkah Pengujian}&\textbf{Hasil yang Diharapkan}&\textbf{Hasil Pengujian}&\textbf{Status}\\ \hline	
	\end{tabular}
	\end{table}	
	
	\item Pengujian pembuatan surat oleh mahasiswa
	\item Pengujian memberikan nomor surat dan \textit{generate} surat
	\item Pengujian mengubah status penandatanganan surat
	\item Pengujian mengubah status pengambilan surat
\end{itemize}